\documentclass{article}
\usepackage{microtype}
\usepackage{mathpazo}
\usepackage{enumitem}
\usepackage{cleveref}
	\crefalias{cex}{Example}

\usepackage{xcolor}
	\definecolor{accol}{rgb}{139,0,0}
	% Defining an accent color

\usepackage{amssymb}
\usepackage{amsthm}
	\newtheoremstyle{accentcolorthm}
		{} % Space above
		{} % Space below
		{} % Body font
		{-2em} % Indent amount
		{\color{accol}} % Theorem head formatting
		{:} % Punctuation following the theorem head
		{.5em} % Space after the theorem head
		{} % Theorem head spec
	
	\theoremstyle{accentcolorthm}
	\newtheorem{cdef}{Definition}[section]

	\theoremstyle{accentcolorthm}
	\newtheorem{cexamp}{Example}[section]



\begin{document}
\section{Classification of Singularities}

Many functions have singularities at $z=0$, but not all singularities are equal.
For example, $(\exp(z)-1)/z$, $z^{-4}$, and $\exp({1/2z})$ all behave differntly
near $z=0$.
We will frequently consider functions in this chapter that are holomorphic in a
disk except at its center (usually because that’s where asingularity lies), and
it will be handy to define the \textbf{punctured disk} with center $z_0$ and radius $R$,
\[ D_\times [z_0, R] := \left\{ z \in \mathbb{C}  : 0 < |z-z_0| < R\right\}  = D[z_0,R] \backslash \{z_0\} \]


\begin{cdef} 
	If $f$ is holomorphic in the punctured disk $D_\times[z_0, R]$ for some $R>0$
	but is not holomorpic at $z=z_0$, then $z_0$ is an
	\textbf{isolated singularity} of $f$. We say that the singularity $z_0$ is
	\begin{enumerate}[label=(\alph*)]
		\item \textbf{removeable} if there exists a function $g$ holomorphic in $D[z_0,R]$
			such that $f=g$ in $D_\times[z_0, R]$, 
		\item a \textbf{pole} if $\displaystyle\lim_{z\to z_0} = \infty$, 
		\item \textbf{essential} if neither a pole or removeable
\end{enumerate}
\end{cdef}
\begin{proof}
	If $A$ then $B$, as seen by Lorem Ipsum Dolor in 1956.
\end{proof}


\begin{cexamp}\label{ex:powers}
	Let $f: \mathbb{C} \backslash {0} \to \mathbb{C}$ be given by $f(z) = (\exp(z)-1)/z$.
	Since 
	\[ \exp(z)-1 = \sum_{k\geq 1}\  \frac{z^k}{k!}, \]
	the function $g:\mathbb{C} \to \mathbb{C}$ is defined by
	\[ g(z) = \sum_{k\geq 0}\  \frac{z^k}{(k+1)!}, \]
	which is entire (because the power series converges in $\mathbb{C}$) agrees with $f$ in
	$\mathbb{C}\backslash \{0\}$. Thus $f$ has a singularity at $0$.

\end{cexamp}






\end{document}

